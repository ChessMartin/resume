\documentclass[
	a4paper, % Uncomment for A4 paper size (default is US letter)
	10pt, % Default font size, can use 10pt, 11pt or 12pt
]{resume} % Use the resume class

%------------------------------------------------
\usepackage{pifont}

\name{Martin Le Formal} % Your name to appear at the top

% You can use the \address command up to 3 times for 3 different addresses or pieces of contact information
% Any new lines (\\) you use in the \address commands will be converted to symbols, so each address will appear as a single line.
\def\CC{{C\nolinebreak[4]\hspace{-.05em}\raisebox{.4ex}{\tiny\bf ++}}}

\address{17 boulevard de Beausejour \\ Paris, 75016} % Main address

\address{(+33) 6 49 89 13 00 \\ martin.le.formal@efrei.net} % Contact information

%----------------------------------------------------------------------------------------

\begin{document}

\begin{rSection}{Education}
	\textbf{EFREI Paris, France} \hfill 2021-2024 (expected) \\
	\textit{\textbf{Master of Engineering in Data Engineering}} \\
	Relevant Coursework: Statistics, Machine Learning, Big Data Frameworks, Cloud Computing

	\textbf{EFREI Paris, France} \hfill 2018-2021 \\
	\textit{\textbf{Bachelor of Engineering in Computer Science}} \\
	Relevant Coursework: Linear Algebra, Probability, Analysis, Algorithmics, Electronics, Databases
\end{rSection}

%----------------------------------------------------------------------------------------
%	WORK EXPERIENCE SECTION
%----------------------------------------------------------------------------------------
\begin{rSection}{Experience}

	\begin{rSubsection}{Embedded Software Engineer - Missile Optics and Electronics}{Apr 2024 - Sep 2024 (current)}{Thales LAS}{Elancourt, France}
    \item Designed an event driven IPC protocol using the inotify Linux API.
    \item Developed a software interface to emulate Rafale commands sent to the Talios POD.
    \item Created a custom Flight Joystick driver based on the Win32 API.
    \item Documented the Talios/Rafale communication system implemented in a 1553 bus.
	\end{rSubsection}

	\begin{rSubsection}{Quant Developer - Automated Market Making - Equities \& Derivatives}{Oct 2022 - Apr 2023}{BNP Paribas CIB}{Paris, France}
		\item Created a log file parser using PySpark and regex for metrics extraction.
		\item Developed an extension to the Python pickle library allowing runtime polymorphisme serialization.
		\item Refactored the market simulation engine launcher used for backtesting automation.
		\item Implemented continuous integration for Python library development through various test cases.
		\item Rewrote the retrieval \& archiving process for production servers configuration files.
	\end{rSubsection}

	\begin{rSubsection}{Quant Developer - Automated Market Making - Equities \& Derivatives}{Jan 2022 - Aug 2022}{BNP Paribas CIB}{Hong Kong SAR, China}
    \item Developed a parser in \CC11 that generates H5 market data files from network packet captures.
		\item Integrated the parser to the Multicast Listener for intraday data generation.
		\item Tested the order management system according to the Thailand Stock Exchange new regulation.
		\item Automated the generation of reports measuring market order success rate.
	\end{rSubsection}

\end{rSection}

\begin{rSection}{Relevant Skills}
	\begin{rSubsection}{\CC20}{}{}{}
    \item Proficient in navigating and managing large codebases using Neovim and LLVM-based LSP configurations.
    \item Experienced with CMake for build configuration and Ninja for efficient build execution.
    \item Identified performance bottlenecks with perf and resolved issues using strace.
    \item Applied knowledge of run-time versus compile-time paradigms and implemented suitable designs.
    \item Designed and implemented thread pools to enhance thread management and improve concurrency.
    \item Conscious of memory allocation for data structures selection.
	\end{rSubsection}
  \begin{rSubsection}{Collaborative Work}{}{}{}
    \item Acquired basic trading knowledge through internships at BNP Paribas and personal research.
    \item Passionate about understanding users' needs to effectively leverage and build upon others’ work.
    \item Enjoy working collaboratively to address challenges and share insights.
    \item Proficient in using Git for efficient collaboration in software development.
  \end{rSubsection}
\end{rSection}


%----------------------------------------------------------------------------------------
%	INTERESTS AND ACTIVITIES
%----------------------------------------------------------------------------------------
\begin{rSection}{Interests and Activities}

	\textbf{Competitive Programming:} ReadyTraderGo (2023), Prologin (2019 \& 2020), LeetCode, Codewars. \\
	\textbf{Chess:} (2/472) University chess championship (2021), Hosted the French university championship (2019).\\
  \textbf{Hobbies:} CppCon \ding{69} Technology related podcast \ding{69} Golf \ding{69} French and Russian literature \ding{69} Cinema.
\end{rSection}

\end{document}
